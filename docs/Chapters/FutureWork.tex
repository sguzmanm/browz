% Chapter Template

\chapter{Future Work} % Main chapter title

\label{ChapterX} % Change X to a consecutive number; for referencing this chapter elsewhere, use \ref{ChapterX}

In this chapter we propose improvements and specialized tasks that could be done after this first stage of the research. First, a more comprehensive search of related work must be done to identify software engineering tasks that has been addressed using static analysis of android apps since 2016. Additionally, this further research can provide more information about the next to be implemented software engineering task (at APK level) in our pipeline, which could be either test cases generation, on-demand documentation, or another one.

At the same time, some effort must be dedicated to fully study the bug taxonomy generated by MDroid+ authors, in order to define more mutation operators or to propose other approaches to identify new possible bugs that could be translated into new mutation operators. Even more important, effort should be devoter to fix the high rate of non-compilable mutants that is generated by MutAPK.

Also, it will be helpful to build a wrapper for MutAPK ( or a new tool ) that is capable of orchestrating the execution of a test suite over the generated mutants. It is important for that solution to offer the possibility of deploying multiple AVD or similar representations and manage them taking into account different challenges as fragmentation, test flakiness, cold starts, etc. \cite{8094439}

As an extension of the research question addressed in this thesis, an extensive study must be done using top applications of the different categories from the Google Play Store, to validate the behavior of MutAPK for more complex applications.
Finally in terms of the implementation of MutAPK, some research effort can be invested in designing a model that improves the location recognition and provides enough information to continue mutating the SMALI representation in the registered times.